% ==== Document Class & Packages =====
\documentclass[12pt,hidelinks]{article}
	\usepackage{amsfonts}
	\usepackage{amsmath}
	\usepackage{amssymb}
	\usepackage{amsthm}
	\usepackage{charter}
	\usepackage{color}
	\usepackage{empheq}
	\usepackage{fancyhdr}
	\usepackage{fancyref}
	\usepackage{fourier}
	\usepackage{geometry}
	\usepackage{graphicx}
	\usepackage{hyperref}
	\usepackage{latexsym}
	\usepackage{listings}
	\usepackage{longfbox}
	\usepackage{marginnote}
	\usepackage{mathrsfs}
	\usepackage{tocloft}
	\usepackage{titletoc}
	\usepackage{varwidth}
	\usepackage{wrapfig}
	\usepackage{xcolor}

	\usepackage{cleveref}
	
	\usepackage[explicit]{titlesec}
	\usepackage[many]{tcolorbox}
	\usepackage{tikz}
	\usepackage{lipsum}
	% \usepackage{lmodern}
	\usepackage[english]{babel}
	
	\usetikzlibrary{calc}
	\tcbuselibrary{theorems}
	\tcbuselibrary{breakable, skins}
	\tcbuselibrary{listings, documentation}
	\geometry{
		a4paper,
		left=33mm,
		right=33mm,
		top=20mm}

% ========= Path to images ============
\graphicspath{{./images/}}

% ============= Macros ================
\newcommand{\fillin}{\underline{\hspace{.75in}}{\;}}
\newcommand{\solution}{\textcolor{mordantred19}{Solution:}}
\setlength{\parindent}{0pt}
\addto{\captionsenglish}{\renewcommand*{\contentsname}{Table of Contents}}
\linespread{1.2}

% =========== New commands ============
% \newcommand{\SolveLP}{\textbf{Solve\_LP}}
\newcommand{\st}{\operatorname{s.t.}}

\newcommand{\N}{\mathbb{N}}
\newcommand{\Z}{\mathbb{Z}}
\newcommand{\R}{\mathbb{R}}

% ======== Footers & Headers ==========
\cfoot{\thepage}
\chead{}\rhead{}\lhead{}
% =====================================
\renewcommand{\thesection}{\arabic{section}}
\newcommand\sectionnumfont{% font specification for the number
	\fontsize{380}{130}\color{myblueii}\selectfont}
\newcommand\sectionnamefont{% font specification for the name "PART"
	\normalfont\color{white}\scshape\small\bfseries }

% ============= Colors ================
% ----- Red -----
\definecolor{mordantred19}{rgb}{0.68, 0.05, 0.0}
% ----- Blue -----
\definecolor{st.patricks blue}{rgb}{0.14, 0.16, 0.48}
\definecolor{teal}{rgb}{0.0, 0.5, 0.5}
\definecolor{beaublue}{rgb}{0.74, 0.83, 0.9}
\definecolor{mybluei}{RGB}{0,173,239}
\definecolor{myblueii}{RGB}{63,200,244}
\definecolor{myblueiii}{RGB}{199,234,253}
% ---- Yellow ----
\definecolor{blond}{rgb}{0.98, 0.94, 0.75}
\definecolor{cream}{rgb}{1.0, 0.99, 0.82}
% ----- Green ------
\definecolor{emerald}{rgb}{0.31, 0.78, 0.47}
\definecolor{darkspringgreen}{rgb}{0.09, 0.45, 0.27}
% ---- White -----
\definecolor{ghostwhite}{rgb}{0.97, 0.97, 1.0}
\definecolor{splashedwhite}{rgb}{1.0, 0.99, 1.0}
% ---- Grey -----
\definecolor{whitesmoke}{rgb}{0.96, 0.96, 0.96}
\definecolor{lightgray}{rgb}{0.92, 0.92, 0.92}
\definecolor{floralwhite}{rgb}{1.0, 0.98, 0.94}

% ========= Part Format ==========
\titleformat{\section}
{\normalfont\huge\filleft}
{}
{20pt}
{\begin{tikzpicture}[remember picture,overlay]
	\fill[myblueiii] 
	(current page.north west) rectangle ([yshift=-13cm]current page.north east);   
\node[
	fill=mybluei,
	text width=2\paperwidth,
	rounded corners=6cm,
	text depth=18cm,
	anchor=center,
	inner sep=0pt] at (current page.north east) (parttop)
	{\thepart};%
\node[
	anchor=south east,
	inner sep=0pt,
	outer sep=0pt] (partnum) at ([xshift=-20pt]parttop.south) 
	{\sectionnumfont\thesection};
\node[
	anchor=south,
	inner sep=0pt] (partname) at ([yshift=2pt]partnum.south)   
	{\sectionnamefont SECTION};
\node[
	anchor=north east,
	align=right,
	inner xsep=0pt] at ([yshift=-0.5cm]partname.east|-partnum.south) 
	{\parbox{.7\textwidth}{\raggedleft#1}};
\end{tikzpicture}%
}

% ========= Hyper Ref ===========
\hypersetup{
	colorlinks,
	linkcolor={red!50!black},
	citecolor={blue!50!black},
	urlcolor={blue!80!black}
}

% ========= Example Boxes =============
\tcbset{
	defstyle/.style={
		fonttitle=\bfseries\upshape, 
		fontupper=\slshape,
		arc=0mm, 
		beamer,
		colback=blue!5!white,
		colframe=blue!75!black},
	theostyle/.style={
		fonttitle=\bfseries\upshape, 
		fontupper=\slshape,
		colback=red!10!white,
		colframe=red!75!black},
	visualstyle/.style={
		height=6.5cm,
		breakable,
		enhanced,
		leftrule=0pt,
		rightrule=0pt,
		bottomrule=0pt,
		outer arc=0pt,
		arc=0pt,
		colframe=mordantred19,
		colback=lightgray,
		attach boxed title to top left,
		boxed title style={
			colback=mordantred19,
			outer arc=0pt,
			arc=0pt,
			top=3pt,
			bottom=3pt,
		},
		fonttitle=\sffamily,},
	discussionstyle/.style={
		height=6.5cm,
		breakable,
		enhanced,
		rightrule=0pt,
		toprule=0pt,
		outer arc=0pt,
		arc=0pt,
		colframe=mordantred19,
		colback=lightgray,
		attach boxed title to top left,
		boxed title style={
			colback=mordantred19,
			outer arc=0pt,
			arc=0pt,
			top=3pt,
			bottom=3pt,
		},
		fonttitle=\sffamily},
	mystyle/.style={
		height=6.5cm,
		breakable,
		enhanced,
		rightrule=0pt,
		leftrule=0pt,
		bottomrule=0pt,
		outer arc=0pt,
		arc=0pt,
		colframe=mordantred19,
		colback=lightgray,
		attach boxed title to top left,
		boxed title style={
			colback=mordantred19,
			outer arc=0pt,
			arc=0pt,
			top=3pt,
			bottom=3pt,
		},
		fonttitle=\sffamily},
	aastyle/.style={
			height=3.5cm,
			enhanced,
			colframe=teal,
			colback=lightgray,
			colbacktitle=floralwhite,
			fonttitle=\bfseries,
			coltitle=black,
		attach boxed title to top center={
	  		yshift=-0.25mm-\tcboxedtitleheight/2,
	   		yshifttext=2mm-\tcboxedtitleheight/2}, 
		boxed title style={boxrule=0.5mm,
			frame code={ \path[tcb fill frame] ([xshift=-4mm]frame.west)
				-- (frame.north west) -- (frame.north east) -- ([xshift=4mm]frame.east)
				-- (frame.south east) -- (frame.south west) -- cycle; },
			interior code={ 
				\path[tcb fill interior] ([xshift=-2mm]interior.west)
				-- (interior.north west) -- (interior.north east)
				-- ([xshift=2mm]interior.east) -- (interior.south east) -- (interior.south west)
				-- cycle;} }
				},
	examstyle/.style={
		height=9.5cm,
		breakable,
		enhanced,
		rightrule=0pt,
		leftrule=0pt,
		bottomrule=0pt,
		outer arc=0pt,
		arc=0pt,
		colframe=mordantred19,
		colback=lightgray,
		attach boxed title to top left,
		boxed title style={
			colback=mordantred19,
			outer arc=0pt,
			arc=0pt,
			top=3pt,
			bottom=3pt,
		},
		fonttitle=\sffamily},
	doc head command={
		interior style={
			fill,
			left color=yellow!20!white, 
			right color=white}},
	doc head environment={
		boxsep=4pt,
		arc=2pt,
		colback=yellow!30!white,
		},
	doclang/environment content=text
}

% ============= Boxes ================
\newtcolorbox[auto counter,number within=section]{example}[1][]{
	mystyle,
	title=Example~\thetcbcounter,
	overlay unbroken and first={
		\path
		let
		\p1=(title.north east),
		\p2=(frame.north east)
		in
		node[anchor=
			west,
			font=\sffamily,
			color=st.patricks blue,
			text width=\x2-\x1] 
		at (title.east) {#1};
	}
}
\newtcolorbox[auto counter,number within=section]{longexample}[1][]{
	examstyle,
	title=Example~\thetcbcounter,
	overlay unbroken and first={
		\path
		let
		\p1=(title.north east),
		\p2=(frame.north east)
		in
		node[anchor=
		west,
		font=\sffamily,
		color=st.patricks blue,
		text width=\x2-\x1] 
		at (title.east) {#1};
	}
}
\newtcolorbox[auto counter,number within=section]{example2}[1][]{
	aastyle,
	title=Example~\thetcbcounter,{}
}
\newtcolorbox[auto counter,number within=section]{discussion}[1][]{
	discussionstyle,
	title=Discussion~\thetcbcounter,
	overlay unbroken and first={
		\path
		let
		\p1=(title.north east),
		\p2=(frame.north east)
		in
		node[anchor=
		west,
		font=\sffamily,
		color=st.patricks blue,
		text width=\x2-\x1] 
		at (title.east) {#1};
	}
}
\newtcolorbox[auto counter,number within=section]{visualization}[1][]{
	visualstyle,
	title=Visualization~\thetcbcounter,
	overlay unbroken and first={
		\path
		let
		\p1=(title.north east),
		\p2=(frame.north east)
		in
		node[anchor=
		west,
		font=\sffamily,
		color=st.patricks blue,
		text width=\x2-\x1] 
		at (title.east) {#1};
	}
}

% --------- Theorems ---------
\newtcbtheorem[number within=subsection,crefname={definition}{definitions}]%
	{Definition}{Definition}{defstyle}{def}%
\newtcbtheorem[use counter from=Definition,crefname={theorem}{theorems}]%
	{Theorem}{Theorem}{theostyle}{theo}
	%
\newtcbtheorem[use counter from=Definition]{theo}{Theorem}%
{
	theorem style=plain,
	enhanced,
	colframe=blue!50!black,
	colback=yellow!20!white,
	coltitle=red!50!black,
	fonttitle=\upshape\bfseries,
	fontupper=\itshape,
	drop fuzzy shadow=blue!50!black!50!white,
	boxrule=0.4pt}{theo}
\newtcbtheorem[use counter from=Definition]{DashedDefinition}{Definition}%
 {
 	enhanced,
 	frame empty,
 	interior empty,
 	colframe=darkspringgreen!50!white,
	coltitle=darkspringgreen!50!black,
	fonttitle=\bfseries,
	colbacktitle=darkspringgreen!15!white,
	borderline={0.5mm}{0mm}{darkspringgreen!15!white},
	borderline={0.5mm}{0mm}{darkspringgreen!50!white,dashed},
	attach boxed title to top center={yshift=-2mm},
	boxed title style={boxrule=0.4pt},
	varwidth boxed title}{theo}
%%%%%%%%%%%%%%%%%%%%%%%%%%%%%%%%%%%%%%%%
\newtcblisting[auto counter,number within=section]{disexam}{
	skin=bicolor,
	colback=white!30!beaublue,
	colbacklower=white,
	colframe=black,
	before skip=\medskipamount,
	after skip=\medskipamount,
	fontlower=\footnotesize,
	listing options={style=tcblatex,texcsstyle=*\color{red!70!black}},}
%%%%%%%%%%%%%%%%%%%%%%%%%%%%%%%%%%%%%%%

\begin{document}
\begin{titlepage}
	\centering % Center everything on the title page
	\scshape % Use small caps for all text on the title page
	\vspace*{1.5\baselineskip} % White space at the top of the page
% ===================
%	Title Section 	
% ===================

	\rule{13cm}{1.6pt}\vspace*{-\baselineskip}\vspace*{2pt} % Thick horizontal rule
	\rule{13cm}{0.4pt} % Thin horizontal rule
	
		\vspace{0.75\baselineskip} % Whitespace above the title
% ========== Title ===============	
	{	\Huge Solve\_LP\\ 
			\vspace{4mm}
		\LARGE	A CLI to solve Linear Programming Problems \\	}
% ======================================
		\vspace{0.75\baselineskip} % Whitespace below the title
	\rule{13cm}{0.4pt}\vspace*{-\baselineskip}\vspace{3.2pt} % Thin horizontal rule
	\rule{13cm}{1.6pt} % Thick horizontal rule
	
		\vspace{1.75\baselineskip} % Whitespace after the title block
% =================
%	Information	
% =================
	% {\large Solve\_LP user guide} \\
	\vfill
Written by: Eduardo dos Santos Teixeira\\
Source code available at: \url{https://github.com/Eduardo281/solve_lp}\\
Critics and suggestions can be sent to: \url{eduardo281.dev@gmail.com}
\end{titlepage}
%%%%%%%%%%%%%%%%%%%%%%%%%%%%%%%%%%%%%%%%%%%%%%%%%%%%%%%%%%%
\tableofcontents
% \vfill
% \small{\noindent \textbf{About This File} \vspace{-3mm}\\
% \noindent \rule{3.3cm}{0.5pt} \\
% This file was created for the benefit of all teachers and students wanting to use Latex for tests/exams/lessons/thesis/articles etc.\\
% The entirety of the contents within this file, and folder, are free for public use.}
\newpage
\newgeometry{
	left=29mm, 
	right=29mm, 
	top=20mm, 
	bottom=15mm
}

%%%%%%%%%%%%%%%%%%%%%%%%%%%%%%%%%%%%%%%%%%%%%%%%%%%%%%%%%%%
\section{Introduction}
\vspace{10.5cm}
	\subsection{CLI Commands}
	\subsubsection{Introduction}
	Solve\_LP is a Command Line Interface (CLI), and so all its utilization is done through commands in a terminal. By now we are assuming users know the basic usage of this kind of tool. In future versions of this manual we will include some more informations to make it more beginner-friendly. In this Section we present all the Solve\_LP arguments.

	\subsubsection{Solve\_LP Commands}
	{\color{mordantred19}
	\begin{verbatim}
		--help/-h
	\end{verbatim}
	} Used to display the general Solve\_LP usage instructions. The text is auto generated by the Python's \texttt{argparse} module using small help instructions provided by the developer, and can be used as a quick reminder of all the Solve\_LP commands.

	{\color{mordantred19}
		\begin{verbatim}
			--file INPUT_FILE_PATH
		\end{verbatim}
	} Solve\_LP is a tool to solve linear programming problems, and so every time it is called, the user needs to specify which is the data file to be input. The \texttt{file} command is used to do it, and it is the only required argument that must be passed every time Solve\_LP is invoked.

	\textbf{Usage example:} 
	\begin{verbatim}
		solve_lp --file DATA_FILE.mat
	\end{verbatim}

	{\color{mordantred19}
		\begin{verbatim}
			--time/-t TOTAL_SOLVING_TIME
		\end{verbatim}
	} Used to specify the total solution time of the optimization process (in general, mixed-integer linear programming solvers use Branch-and-Cut methods) in seconds.
	
	\textbf{Usage example:} 
	\begin{verbatim}
		solve_lp --file DATA_FILE.mat --time 30
	\end{verbatim}

	{\color{mordantred19}
		\begin{verbatim}
			--solver/-s SOLVER_NAME
		\end{verbatim}
	} Used to specify which optimization solver will be called by Solve\_LP to tackle the input problem. The available solvers depend upon the modeling language used, and for each language it can change for different reasons. We will present a complete list of possible combinations of solver and modeling language in another Section of this manual.
	
	\textbf{Usage example:} 
	\begin{verbatim}
		solve_lp --file DATA_FILE.mat --solver CBC
	\end{verbatim}

	{\color{mordantred19}
	\begin{verbatim}
		--gap/-g REL_MIP_GAP_VALUE
	\end{verbatim}
	} Integer programming problems may involve very deep Branch-and-Cut trees, which may cause the solution procedure to be very time consuming. When this is the case, the solver will try to reach sufficiently close to an optimal solution (in practice, almost any computational mathematics problem is solved approximately). The gap represents this tolerance level accepted by the solver to consider that a solution is optimal.
	
	\textbf{Usage example:} 
	\begin{verbatim}
		solve_lp --file DATA_FILE.lp --solver CBC --time 600 --gap 0.01
	\end{verbatim}

	{\color{mordantred19}
	\begin{verbatim}
		--digits/-d DIGITS_ON_SOLUTION
	\end{verbatim}
	} Since computational mathematics problems are solved using approximately values, it may cause the final solutions obtained to be shown as approximately numbers, and so the user may get $x_1 = 0.99999$ instead of $x_1 = 1$. The \texttt{digits} command is used to round the values by specifying the maximum number of digits to be considered.
	
	\textbf{Usage example:} 
	\begin{verbatim}
		solve_lp --file DATA_FILE.json --digits 5 --time 60
	\end{verbatim}

	{\color{mordantred19}
	\begin{verbatim}
		--threads/-r NUM_THREADS
	\end{verbatim}
	} Some optimization solvers can perform parallel solution procedures, using more than a thread to tackle a problem by applying Branch-and-Cut, heuristics and maybe other solution strategies. The \texttt{threads} command is used to specify the maximum number of threads the solver is allowed to use, assuming the specified solver can use more than only one.
	
	\textbf{Usage example:} 
	\begin{verbatim}
		solve_lp --file DATA_FILE.lp --time 120 --threads 4
	\end{verbatim}

	{\color{mordantred19}
	\begin{verbatim}
		--modeling-language
	\end{verbatim}
	} Used to specify which modeling language will be used by Solve\_LP among the ones available, which we describe in a later Section. Each modeling language have its own pros and cons, and sometimes this will not have any impact on the overall solution process. Experienced users may find this useful to test/apply some particular aspect of a desirable modeling language.
	
	\textbf{Usage example:} 
	\begin{verbatim}
		solve_lp --file DATA.mat --modeling-language OR-TOOLS --solver CBC
	\end{verbatim}

	{\color{mordantred19}
	\begin{verbatim}
		--zeros/--no-zeros
	\end{verbatim}
	} The command \texttt{zeros} is used to display all the variables in the final result, even the ones that have zero value. In cases the expected result is mostly made of zero values, the user can apply \texttt{no-zeros} to avoid an unnecessarily large result print.
	
	\textbf{Usage example:} 
	\begin{verbatim}
		solve_lp --file DATA_FILE.mat --time 300 --zeros
		solve_lp --file DATA_FILE.lp --solver SCIP --no-zeros
	\end{verbatim}

	{\color{mordantred19}
	\begin{verbatim}
		--log/--no-log
	\end{verbatim}
	} Optimization solvers commonly offer the option of printing a log of the solution procedure. The commands \texttt{log} and \texttt{no-log} are used to control if the user want to see this log or not.
	
	\textbf{Usage example:} 
	\begin{verbatim}
		solve_lp --file DATA_FILE.mat --time 900 --solver CBC --log
		solve_lp --file DATA_FILE.lp --modeling-language GUROBIPY --no-log
	\end{verbatim}

	{\color{mordantred19}
	\begin{verbatim}
		--verbose/--no-verbose
	\end{verbatim}
	} The pair of commands \texttt{verbose} and \texttt{no-verbose} are used to show or not to show the Solve\_LP solving process steps log. We emphasize this is not the solver log, but a small log of the steps described by Solve\_LP as it read the input data file, process it and solve the problem.
	
	\textbf{Usage example:} 
	\begin{verbatim}
		solve_lp --file DATA_FILE.json --verbose --no-log
		solve_lp --file DATA_FILE.mps --time 30 --no-verbose --no-zeros
	\end{verbatim}

	{\color{mordantred19}
	\begin{verbatim}
		--export-solution-json JSON_FILEPATH
	\end{verbatim}
	} Solve\_LP offers the option of exporting the solving results in a .json file through the command \texttt{export-solution-json}.
	
	\textbf{Usage example:} 
	\begin{verbatim}
		solve_lp --file DATA.mat --export-solution-json SOLUTION.json
	\end{verbatim}

	{\color{mordantred19}
	\begin{verbatim}
		--export-mps-model MPS_FILEPATH
	\end{verbatim}
	} Used to export the input model in a .mps file.
	
	\textbf{Usage example:} 
	\begin{verbatim}
		solve_lp --file DATA.mat --export-mps-model MODELFILE.mps
	\end{verbatim}

	{\color{mordantred19}
	\begin{verbatim}
		--export-lp-model LP_FILEPATH
	\end{verbatim}
	} Used to export the input model in a .lp file.
	
	\textbf{Usage example:} 
	\begin{verbatim}
		solve_lp --file DATA.mat --export-lp-model MODELFILE.lp
	\end{verbatim}

	{\color{mordantred19}
	\begin{verbatim}
		--export-logfile LOGFILE_PATH
	\end{verbatim}
	} Used to export the solver log in a text file. There is not standard format to this file, but we suggest to use .log as file type.

	\textbf{Important Note:} This functionality was implemented by re-directing the solver log output to the logfile ``manually''. By the time this manual was written we did not find any error in this process, but this is still an experimental option.
	
	\textbf{Usage example:} 
	\begin{verbatim}
		solve_lp --file DATA.lp --export-logfile LOGFILE.log
	\end{verbatim}

	{\color{mordantred19}
	\begin{verbatim}
		--matrix-input
	\end{verbatim}
	} Tells Solve\_LP to assume the input data file is in .mat format.

	\textbf{Usage example:} 
	\begin{verbatim}
		solve_lp --file DATA_FILE.mat --matrix-input
	\end{verbatim}

	{\color{mordantred19}
	\begin{verbatim}
		--json-input
	\end{verbatim}
	} Tells Solve\_LP to assume the input data file is in .json format.

	\textbf{Usage example:} 
	\begin{verbatim}
		solve_lp --file DATA_FILE.json --json-input
	\end{verbatim}

	{\color{mordantred19}
	\begin{verbatim}
		--mps-input
	\end{verbatim}
	} Tells Solve\_LP to assume the input data file is in .mps format.

	\textbf{Usage example:} 
	\begin{verbatim}
		solve_lp --file DATA_FILE.mps --mps-input
	\end{verbatim}

	{\color{mordantred19}
	\begin{verbatim}
		--lp-input
	\end{verbatim}
	} Tells Solve\_LP to assume the input data file is in .lp format.

	\textbf{Usage example:} 
	\begin{verbatim}
		solve_lp --file DATA_FILE.lp --lp-input
	\end{verbatim}

	{\color{mordantred19}
	\begin{verbatim}
		--infer-input
	\end{verbatim}
	} Solve\_LP can analyze the input file and try to detect its format without the need of indicating it explicitly. This process can lead to a few slower data reading process, but it is a good option in general cases.

	\textbf{Important Note:} Although Solve\_LP can recognize .mps and .lp files, the optimization solver only solve these types of files if they have the right file extension. In other words, if the user input an mps file named ``my\_data'' to Solve\_LP, it will recognize it is a .mps file, but the optimization solver will not solve it because it was not able to identify it. The user should rename it to ``my\_data.mps''.

	\textbf{Usage example:} 
	\begin{verbatim}
		solve_lp --file DATA_FILE.mat --infer-input
	\end{verbatim}

	{\color{mordantred19}
	\begin{verbatim}
		--skip-solve
	\end{verbatim}
	} Used to only read the input data and execute model export commands, but not to solve the problem. Useful to perform conversions between different model formats

	\textbf{Usage example:} 
	\begin{verbatim}
		solve_lp --file DATA.mat --skip-solve --export-mps-model MODELFILE.mps
	\end{verbatim}

	{\color{mordantred19}
	\begin{verbatim}
		--memory-limit MEMORY_LIMIT_IN_GB
	\end{verbatim}
	} Allows to impose a limit on the total amount of memory available to the solving process. Can only be used when the modeling language is set to GUROBIPY.

	\textbf{Usage example:} 
	\begin{verbatim}
		solve_lp --file DATA.mat --modeling-language GUROBIPY --memory-limit 6
	\end{verbatim}

	{\color{mordantred19}
	\begin{verbatim}
		--nodefile-start-size NODEFILE_START_SIZE_IN_GB
	\end{verbatim}
	} Used to store the branch-and-cut tree nodes compressed in the disk, as an way to reduce RAM memory usage. When the total of memory used by the solver reaches the imposed limit value, the solver starts to write the nodes to the disk. Can only be used when the modeling language is set to GUROBIPY

	\textbf{Usage example:} 
	\begin{verbatim}
		solve_lp --file DATA.mat --modeling-language GUROBIPY
		--nodefile-start-size 0.5
	\end{verbatim}

	% \vspace{-1.5mm}
\newpage

%%%%%%%%%%%%%%%%%%%%%%%%%%%%%%%%%%%%%%%%%%%%%%%%%%%%%%%%%%%
\section{Using Solve\_LP}
\label{section:usage}
\vspace{10.5cm}
	\subsection{Mathematical Notations}
		Solve\_LP receives as input the matricial representation of a Linear Programming Problem, following the notations:

		% \begin{itemize}
			$\bullet$ $\mathbf{x}$ represents the decision variables, where:
			$$\mathbf{x} = [x_1, \quad x_2, \quad x_3, \quad ..., \quad x_n]$$
			$\bullet$ $\mathbf{c}$ represents the costs vector, where:
			$$\mathbf{c} = [c_1, \quad c_2, \quad c_3, \quad ..., \quad c_n]$$
			$\bullet$ $\mathbf{b}$ represents the resources vector (a.k.a. the RHS - Right Hand Side - vector):
			$$\mathbf{b} = [b_1, \quad b_2, \quad b_3, \quad ..., \quad b_m]$$
			$\bullet$ $\mathbf{A}$ represents the coefficients matrix, given by:
			$$\mathbf{A} = \left[\begin{array}{ccccc}
				a_{11} & a_{12} & a_{13} & \cdots & a_{1n} \\
				a_{21} & a_{22} & a_{23} & \cdots & a_{2n} \\
				a_{31} & a_{32} & a_{33} & \cdots & a_{3n} \\
				\vdots & \vdots & \vdots & \ddots & \vdots \\
				a_{m1} & a_{m2} & a_{m3} & \cdots & a_{mn} \\
			\end{array}\right] $$				
		% \end{itemize}

		In which follows, it's considered as a linear programming problem in \emph{standard format} a minimization problem where all the constraints are equalities and all the variables are greater or equal to zero and do not have any upper bound. Using the formal mathematical notation, it can be generically represented as in \eqref{LP_STD}.
		\begin{equation}\label{LP_STD}
			\begin{array}{llc}
				\min                & &  \mathbf{c}^T \mathbf{x}            \\
				\operatorname{s.t.} & &  \mathbf{A}\mathbf{x} = \mathbf{b}  \\
									& &   \mathbf{x} \geqslant 0
			\end{array}
		\end{equation}

		An example of problem in this format is given in \eqref{LP_1_0_0}:

			\begin{equation}\label{LP_1_0_0}
				\begin{array}{llrcrcrcrcrcr}
				\min  & & 7x_1  & + & 9x_2  & + & 8x_3  & + & 5x_4  & + & 4x_5 &           &    \\
				\st   & & 9x_1  & + & 4x_2  & + & 6x_3  & + & 6x_4  & + & 5x_5 & =         & 42 \\
					  & & 4x_1  & + & 8x_2  & + & 3x_3  & + & 4x_4  & + & 1x_5 & =         & 62 \\
					  & & 8x_1  & + & 5x_2  & + & 5x_3  & + & 4x_4  & + & 7x_5 & =         & 46 \\
					  & &  x_1, &   &  x_2, &   &  x_3, &   &  x_4, &   &  x_5 & \geqslant & 0
				\end{array}
				\tag{LP1}
			\end{equation}

		It is not hard to solve \eqref{LP_1_0_0} to optimality and find that its optimal value is:
		$$x_1 = 0, x_2 = \frac{20}{3}, x_3 = 0, x_4 = 2, x_5 = \frac{2}{3}.$$

		Solve\_LP allows users to input problems in more general formats, including inequality constraints and redefining the variables lower and upper bounds. Problems input defined only by the parameters $\mathbf{A}$, $\mathbf{b}$ and $\mathbf{c}$ will be interpreted as problems in the standard format.
	
	\subsection{File Formats}
		There are different ways to input data to Solve\_LP, some of them are classical linear optimization file formats, and others were developed to make Solve\_LP friendly to solve small problems, and offer human-readable ways to work with input data.
		\subsubsection{Matrix Format}
			The simplest format users can adopt to input data, the matrix format was especially developed to offer Solve\_LP an easy and user-friendly way to input models and data using a vetorial notation similar to other mathematical softwares like SciLab and Octave.

		\subsubsection{Json Format}
			The json file format is very important in many different contexts when working with technology, particularly in web development. Users can pass data to Solve\_LP using .json files with notations similar to the ones used in the matrix format.

		\subsubsection{LP Format}
			An user-friendly file format commonly accepeted by commercial and open source Optimization solvers, based on record the model and its data like the original mathematical representation. 

		\subsubsection{MPS Format}
			A classical Linear Programming file format used to store models and its data. Not an user-friendly format and not the most used format to work with Optimization nowadays. More information can be found at: 
			
			\url{https://en.wikipedia.org/wiki/MPS_(format)}
\newpage


%%%%%%%%%%%%%%%%%%%%%%%%%%%%%%%%%%%%%%%%%%%%%%%%%%%%%%%%%%%
\section{Solvers and Modeling Languages}
\vspace{10.5cm}
	\subsection{Modeling Languages}
		In its present version Solve\_LP can process the input data using three different modeling languages:
		\begin{itemize}
			\item OR-Tools: Open source project developed by Google that allows to solve continuous linear programming problems, as well as integer/mixed integer linear programming problems, constraint programming problems and includes some specialized solvers for some classic problems. In the case of Solve\_LP, we only use its linear programming capabilities.
			\item Gurobipy: Modeling language developed by the Gurobi Optimization to be used along with the Gurobi Optimizer. It allows to model continuous and mixed integer linear and quadratic programming problems. In this project we are only interested in the linear programming features.
			\item Python-MIP: Open source modeling language developed as an academic initiative on Brazil that became part of the COIN-OR Foundation in 2019. Compatible with PyPy and with support to two important linear programming solvers, it is one of the options available on Solve\_LP.
		\end{itemize}

	\subsubsection{Useful Links}
		The user interested in read more about the modeling languages may find the following links useful:

		$\bullet$ \url{https://developers.google.com/optimization}

		Google developers page for OR-Tools.

		$\bullet$ \url{https://github.com/google/or-tools}

		OR-Tools's GitHub page.

		$\bullet$ \url{https://en.wikipedia.org/wiki/OR-Tools}

		OR-Tools's Wikipedia page.

		$\bullet$ \url{https://www.gurobi.com/}

		Gurobi official website.

		$\bullet$ \url{https://github.com/Gurobi}

		Gurobi's GitHub profile.

		$\bullet$ \url{https://www.python-mip.com/}

		Python-MIP's official page.

		$\bullet$ \url{https://github.com/coin-or/python-mip}

		Python-MIP's GitHub page.

	\subsection{Available Solvers}

	Each modeling language offers support to different optimization solvers. In this Sub-Section we present a brief description about them, and describe which solver can be used in combination with each modeling language. Most part of the solvers have easy-to-find webpages, where the owners/developers may clarify the less trivial doubts of the user. Please remember that part of the solvers require licenses to work, and that commercial uses of the solvers may required permission, even in the case of free for academic purposes solvers.

	\subsubsection{OR-Tools}

		The OR-Tools modeling language offers support to a lot of different linear optimization solvers, which we describe below:

		\begin{itemize}
			\item CLP: The COIN-OR Linear Programming solver. An open source project developed to solve continuous linear programming problems.
			\item CBC: The COIN-OR Branch-and-Cut solver. An open source project developed to solve binary/integer/mixed-integer linear programming problems.
			\item GLOP: An open source linear programming solver developed by Google. Solves only continuous problems.
			\item BOP: A free binary optimization solver.
			\item SAT or CP\_SAT: The Google's CP-SAT solver. A constraint programming solver that uses SAT (satisfiability) methods.
			\item SCIP: A free for academic, non-commercial purposes solver for linear programming problems, continuos, integer or mixed-integer.
			\item GUROBI: A solver developed by the Gurobi Optimization group. Can be used to solve continuous, integer or mixed-integer linear programming problems.
			\item CPLEX: A solver developed by IBM. Can be used to solve continuous, integer or mixed-integer linear programming problems.
			\item XPRESS: A solver developed by FICO. Can be used to solve continuous, integer or mixed-integer linear programming problems.
			\item GLPK: The GNU Linear Programming Kit, developed to solve both continuous and mixed-integer linear programming problems.
			\item PDLP: A solver developed by Google that can be used to solve linear programming problems by using a Primal-Dual Hybrid Gradient method.
		\end{itemize}

	\subsubsection{Gurobipy}

		The Gurobipy modeling language was especially developed to be used together with the Gurobi solver. It has some particular interesting tools users can utilize, as described in the command line arguments Section above.

	\subsubsection{Python-MIP}

		At the moment this manual was written, the Python-MIP official site was showing a message saying that Python-MIP is currently integrated with the C libraries of COIN-OR CBC solver and the commercial solver Gurobi, and that other solvers will be supported soon. The user may refer to the site to see if the support to any different solver was already made available in the last days. 

\newpage
\end{document}
