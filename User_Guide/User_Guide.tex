% ==== Document Class & Packages =====
\documentclass[12pt,hidelinks]{article}
	\usepackage{amsfonts}
	\usepackage{amsmath}
	\usepackage{amssymb}
	\usepackage{amsthm}
	\usepackage{charter}
	\usepackage{color}
	\usepackage{empheq}
	\usepackage{fancyhdr}
	\usepackage{fancyref}
	\usepackage{fourier}
	\usepackage{geometry}
	\usepackage{graphicx}
	\usepackage{hyperref}
	\usepackage{latexsym}
	\usepackage{listings}
	\usepackage{longfbox}
	\usepackage{marginnote}
	\usepackage{mathrsfs}
	\usepackage{tocloft}
	\usepackage{titletoc}
	\usepackage{varwidth}
	\usepackage{wrapfig}
	\usepackage{xcolor}

	\usepackage{cleveref}
	
	\usepackage[explicit]{titlesec}
	\usepackage[many]{tcolorbox}
	\usepackage{tikz}
	\usepackage{lipsum}
	% \usepackage{lmodern}
	\usepackage[english]{babel}
	
	\usetikzlibrary{calc}
	\tcbuselibrary{theorems}
	\tcbuselibrary{breakable, skins}
	\tcbuselibrary{listings, documentation}
	\geometry{
		a4paper,
		left=33mm,
		right=33mm,
		top=20mm}

% ========= Path to images ============
\graphicspath{{./images/}}

% ============= Macros ================
\newcommand{\fillin}{\underline{\hspace{.75in}}{\;}}
\newcommand{\solution}{\textcolor{mordantred19}{Solution:}}
\setlength{\parindent}{0pt}
\addto{\captionsenglish}{\renewcommand*{\contentsname}{Table of Contents}}
\linespread{1.2}

% =========== New commands ============
\newcommand{\SolveLP}{\textbf{Solve\_LP}}
\newcommand{\st}{\operatorname{s.t.}}

\newcommand{\N}{\mathbb{N}}
\newcommand{\Z}{\mathbb{Z}}
\newcommand{\R}{\mathbb{R}}

% ======== Footers & Headers ==========
\cfoot{\thepage}
\chead{}\rhead{}\lhead{}
% =====================================
\renewcommand{\thesection}{\arabic{section}}
\newcommand\sectionnumfont{% font specification for the number
	\fontsize{380}{130}\color{myblueii}\selectfont}
\newcommand\sectionnamefont{% font specification for the name "PART"
	\normalfont\color{white}\scshape\small\bfseries }

% ============= Colors ================
% ----- Red -----
\definecolor{mordantred19}{rgb}{0.68, 0.05, 0.0}
% ----- Blue -----
\definecolor{st.patricks blue}{rgb}{0.14, 0.16, 0.48}
\definecolor{teal}{rgb}{0.0, 0.5, 0.5}
\definecolor{beaublue}{rgb}{0.74, 0.83, 0.9}
\definecolor{mybluei}{RGB}{0,173,239}
\definecolor{myblueii}{RGB}{63,200,244}
\definecolor{myblueiii}{RGB}{199,234,253}
% ---- Yellow ----
\definecolor{blond}{rgb}{0.98, 0.94, 0.75}
\definecolor{cream}{rgb}{1.0, 0.99, 0.82}
% ----- Green ------
\definecolor{emerald}{rgb}{0.31, 0.78, 0.47}
\definecolor{darkspringgreen}{rgb}{0.09, 0.45, 0.27}
% ---- White -----
\definecolor{ghostwhite}{rgb}{0.97, 0.97, 1.0}
\definecolor{splashedwhite}{rgb}{1.0, 0.99, 1.0}
% ---- Grey -----
\definecolor{whitesmoke}{rgb}{0.96, 0.96, 0.96}
\definecolor{lightgray}{rgb}{0.92, 0.92, 0.92}
\definecolor{floralwhite}{rgb}{1.0, 0.98, 0.94}

% ========= Part Format ==========
\titleformat{\section}
{\normalfont\huge\filleft}
{}
{20pt}
{\begin{tikzpicture}[remember picture,overlay]
	\fill[myblueiii] 
	(current page.north west) rectangle ([yshift=-13cm]current page.north east);   
\node[
	fill=mybluei,
	text width=2\paperwidth,
	rounded corners=6cm,
	text depth=18cm,
	anchor=center,
	inner sep=0pt] at (current page.north east) (parttop)
	{\thepart};%
\node[
	anchor=south east,
	inner sep=0pt,
	outer sep=0pt] (partnum) at ([xshift=-20pt]parttop.south) 
	{\sectionnumfont\thesection};
\node[
	anchor=south,
	inner sep=0pt] (partname) at ([yshift=2pt]partnum.south)   
	{\sectionnamefont SECTION};
\node[
	anchor=north east,
	align=right,
	inner xsep=0pt] at ([yshift=-0.5cm]partname.east|-partnum.south) 
	{\parbox{.7\textwidth}{\raggedleft#1}};
\end{tikzpicture}%
}

% ========= Hyper Ref ===========
\hypersetup{
	colorlinks,
	linkcolor={red!50!black},
	citecolor={blue!50!black},
	urlcolor={blue!80!black}
}

% ========= Example Boxes =============
\tcbset{
	defstyle/.style={
		fonttitle=\bfseries\upshape, 
		fontupper=\slshape,
		arc=0mm, 
		beamer,
		colback=blue!5!white,
		colframe=blue!75!black},
	theostyle/.style={
		fonttitle=\bfseries\upshape, 
		fontupper=\slshape,
		colback=red!10!white,
		colframe=red!75!black},
	visualstyle/.style={
		height=6.5cm,
		breakable,
		enhanced,
		leftrule=0pt,
		rightrule=0pt,
		bottomrule=0pt,
		outer arc=0pt,
		arc=0pt,
		colframe=mordantred19,
		colback=lightgray,
		attach boxed title to top left,
		boxed title style={
			colback=mordantred19,
			outer arc=0pt,
			arc=0pt,
			top=3pt,
			bottom=3pt,
		},
		fonttitle=\sffamily,},
	discussionstyle/.style={
		height=6.5cm,
		breakable,
		enhanced,
		rightrule=0pt,
		toprule=0pt,
		outer arc=0pt,
		arc=0pt,
		colframe=mordantred19,
		colback=lightgray,
		attach boxed title to top left,
		boxed title style={
			colback=mordantred19,
			outer arc=0pt,
			arc=0pt,
			top=3pt,
			bottom=3pt,
		},
		fonttitle=\sffamily},
	mystyle/.style={
		height=6.5cm,
		breakable,
		enhanced,
		rightrule=0pt,
		leftrule=0pt,
		bottomrule=0pt,
		outer arc=0pt,
		arc=0pt,
		colframe=mordantred19,
		colback=lightgray,
		attach boxed title to top left,
		boxed title style={
			colback=mordantred19,
			outer arc=0pt,
			arc=0pt,
			top=3pt,
			bottom=3pt,
		},
		fonttitle=\sffamily},
	aastyle/.style={
			height=3.5cm,
			enhanced,
			colframe=teal,
			colback=lightgray,
			colbacktitle=floralwhite,
			fonttitle=\bfseries,
			coltitle=black,
		attach boxed title to top center={
	  		yshift=-0.25mm-\tcboxedtitleheight/2,
	   		yshifttext=2mm-\tcboxedtitleheight/2}, 
		boxed title style={boxrule=0.5mm,
			frame code={ \path[tcb fill frame] ([xshift=-4mm]frame.west)
				-- (frame.north west) -- (frame.north east) -- ([xshift=4mm]frame.east)
				-- (frame.south east) -- (frame.south west) -- cycle; },
			interior code={ 
				\path[tcb fill interior] ([xshift=-2mm]interior.west)
				-- (interior.north west) -- (interior.north east)
				-- ([xshift=2mm]interior.east) -- (interior.south east) -- (interior.south west)
				-- cycle;} }
				},
	examstyle/.style={
		height=9.5cm,
		breakable,
		enhanced,
		rightrule=0pt,
		leftrule=0pt,
		bottomrule=0pt,
		outer arc=0pt,
		arc=0pt,
		colframe=mordantred19,
		colback=lightgray,
		attach boxed title to top left,
		boxed title style={
			colback=mordantred19,
			outer arc=0pt,
			arc=0pt,
			top=3pt,
			bottom=3pt,
		},
		fonttitle=\sffamily},
	doc head command={
		interior style={
			fill,
			left color=yellow!20!white, 
			right color=white}},
	doc head environment={
		boxsep=4pt,
		arc=2pt,
		colback=yellow!30!white,
		},
	doclang/environment content=text
}

% ============= Boxes ================
\newtcolorbox[auto counter,number within=section]{example}[1][]{
	mystyle,
	title=Example~\thetcbcounter,
	overlay unbroken and first={
		\path
		let
		\p1=(title.north east),
		\p2=(frame.north east)
		in
		node[anchor=
			west,
			font=\sffamily,
			color=st.patricks blue,
			text width=\x2-\x1] 
		at (title.east) {#1};
	}
}
\newtcolorbox[auto counter,number within=section]{longexample}[1][]{
	examstyle,
	title=Example~\thetcbcounter,
	overlay unbroken and first={
		\path
		let
		\p1=(title.north east),
		\p2=(frame.north east)
		in
		node[anchor=
		west,
		font=\sffamily,
		color=st.patricks blue,
		text width=\x2-\x1] 
		at (title.east) {#1};
	}
}
\newtcolorbox[auto counter,number within=section]{example2}[1][]{
	aastyle,
	title=Example~\thetcbcounter,{}
}
\newtcolorbox[auto counter,number within=section]{discussion}[1][]{
	discussionstyle,
	title=Discussion~\thetcbcounter,
	overlay unbroken and first={
		\path
		let
		\p1=(title.north east),
		\p2=(frame.north east)
		in
		node[anchor=
		west,
		font=\sffamily,
		color=st.patricks blue,
		text width=\x2-\x1] 
		at (title.east) {#1};
	}
}
\newtcolorbox[auto counter,number within=section]{visualization}[1][]{
	visualstyle,
	title=Visualization~\thetcbcounter,
	overlay unbroken and first={
		\path
		let
		\p1=(title.north east),
		\p2=(frame.north east)
		in
		node[anchor=
		west,
		font=\sffamily,
		color=st.patricks blue,
		text width=\x2-\x1] 
		at (title.east) {#1};
	}
}

% --------- Theorems ---------
\newtcbtheorem[number within=subsection,crefname={definition}{definitions}]%
	{Definition}{Definition}{defstyle}{def}%
\newtcbtheorem[use counter from=Definition,crefname={theorem}{theorems}]%
	{Theorem}{Theorem}{theostyle}{theo}
	%
\newtcbtheorem[use counter from=Definition]{theo}{Theorem}%
{
	theorem style=plain,
	enhanced,
	colframe=blue!50!black,
	colback=yellow!20!white,
	coltitle=red!50!black,
	fonttitle=\upshape\bfseries,
	fontupper=\itshape,
	drop fuzzy shadow=blue!50!black!50!white,
	boxrule=0.4pt}{theo}
\newtcbtheorem[use counter from=Definition]{DashedDefinition}{Definition}%
 {
 	enhanced,
 	frame empty,
 	interior empty,
 	colframe=darkspringgreen!50!white,
	coltitle=darkspringgreen!50!black,
	fonttitle=\bfseries,
	colbacktitle=darkspringgreen!15!white,
	borderline={0.5mm}{0mm}{darkspringgreen!15!white},
	borderline={0.5mm}{0mm}{darkspringgreen!50!white,dashed},
	attach boxed title to top center={yshift=-2mm},
	boxed title style={boxrule=0.4pt},
	varwidth boxed title}{theo}
%%%%%%%%%%%%%%%%%%%%%%%%%%%%%%%%%%%%%%%%
\newtcblisting[auto counter,number within=section]{disexam}{
	skin=bicolor,
	colback=white!30!beaublue,
	colbacklower=white,
	colframe=black,
	before skip=\medskipamount,
	after skip=\medskipamount,
	fontlower=\footnotesize,
	listing options={style=tcblatex,texcsstyle=*\color{red!70!black}},}
%%%%%%%%%%%%%%%%%%%%%%%%%%%%%%%%%%%%%%%

\begin{document}
\begin{titlepage}
	\centering % Center everything on the title page
	\scshape % Use small caps for all text on the title page
	\vspace*{1.5\baselineskip} % White space at the top of the page
% ===================
%	Title Section 	
% ===================

	\rule{13cm}{1.6pt}\vspace*{-\baselineskip}\vspace*{2pt} % Thick horizontal rule
	\rule{13cm}{0.4pt} % Thin horizontal rule
	
		\vspace{0.75\baselineskip} % Whitespace above the title
% ========== Title ===============	
	{	\Huge Solve\_LP\\ 
			\vspace{4mm}
		\LARGE	A CLI to solve Linear Programming Problems \\	}
% ======================================
		\vspace{0.75\baselineskip} % Whitespace below the title
	\rule{13cm}{0.4pt}\vspace*{-\baselineskip}\vspace{3.2pt} % Thin horizontal rule
	\rule{13cm}{1.6pt} % Thick horizontal rule
	
		\vspace{1.75\baselineskip} % Whitespace after the title block
% =================
%	Information	
% =================
	{\large A complete Solve\_LP user guide \\
		\vspace*{1.2\baselineskip}
	e-mail@domain.com} \\
	\vfill
Written by: Eduardo dos Santos Teixeira\\
Source code available at: \url{https://github.com/Eduardo281/solve_lp}\\
Critics and suggestions can be sent to: \url{e-mail@domain.com}
\end{titlepage}
%%%%%%%%%%%%%%%%%%%%%%%%%%%%%%%%%%%%%%%%%%%%%%%%%%%%%%%%%%%
\tableofcontents
% \vfill
% \small{\noindent \textbf{About This File} \vspace{-3mm}\\
% \noindent \rule{3.3cm}{0.5pt} \\
% This file was created for the benefit of all teachers and students wanting to use Latex for tests/exams/lessons/thesis/articles etc.\\
% The entirety of the contents within this file, and folder, are free for public use.}
\newpage
\newgeometry{
	left=29mm, 
	right=29mm, 
	top=20mm, 
	bottom=15mm
}

%%%%%%%%%%%%%%%%%%%%%%%%%%%%%%%%%%%%%%%%%%%%%%%%%%%%%%%%%%%
\section{Introduction}
\vspace{10.5cm}
	\subsection{General Items}
	{\color{mordantred19}
	\begin{verbatim}
		--help/-h
	\end{verbatim}
	} Used to display the general \SolveLP usage instructions. The text is auto generated by the Python's \texttt{argparse} module, and can be used as a quick reminder of all the \SolveLP commands.

	{\color{mordantred19}
		\begin{verbatim}
			--file INPUT_FILE_PATH
		\end{verbatim}
	} \SolveLP is a tool to solve linear programming problems, and so every time it is called, the user needs to specify which is the data file to be input. The \texttt{file} command is used to do it, and it is the only required argument that must be passed every time \SolveLP is invoked.

	\textbf{Usage example:} 
	\begin{verbatim}
		solve_lp --file DATA_FILE.mat
	\end{verbatim}

	{\color{mordantred19}
		\begin{verbatim}
			--time/-t TOTAL_SOLVING_TIME
		\end{verbatim}
	} Used to specify the total solution time of the optimization process (in general, the Branch-and-Cut method) in seconds
	
	\textbf{Usage example:} 
	\begin{verbatim}
		solve_lp --file DATA_FILE.mat --time 30
	\end{verbatim}

	{\color{mordantred19}
		\begin{verbatim}
			--solver/-s SOLVER_NAME
		\end{verbatim}
	} Used to specify which optimization solver will be called by \SolveLP. The available solvers depend of the modeling language used, and for each language it can change for different reasons. We will present a complete list of possible combinations of solver and modeling language in another section of this manual.
	
	\textbf{Usage example:} 
	\begin{verbatim}
		solve_lp --file DATA_FILE.mat --solver CBC
	\end{verbatim}

	{\color{mordantred19}
	\begin{verbatim}
		--gap/-g REL_MIP_GAP_VALUE
	\end{verbatim}
	} Integer programming problems may involve very deep Branch-and-Cut trees, which may cause the solution procedure to be very time consuming. When this is the case, the solver will try to reach sufficiently close to optimal solution (in practice, any computational mathematics problem is solved this way). The gap represents this tolerance level accepted by the solver to consider that a solution is optimal.
	
	\textbf{Usage example:} 
	\begin{verbatim}
		solve_lp --file DATA_FILE.lp --solver CBC --time 600 --gap 0.01
	\end{verbatim}

	{\color{mordantred19}
	\begin{verbatim}
		--digits/-d DIGITS_ON_SOLUTION
	\end{verbatim}
	} Since computational mathematics problems are solved using approximately values, it may cause the final solutions obtained to be shown as approximately numbers, and so the user may get $x_1 = 0.99999$ instead of $x_1 = 1$. The \texttt{digits} command is used to round the values by specifying the maximum number of digits to be considered.
	
	\textbf{Usage example:} 
	\begin{verbatim}
		solve_lp --file DATA_FILE.json --digits 5 --time 60
	\end{verbatim}

	{\color{mordantred19}
	\begin{verbatim}
		--threads/-r NUM_THREADS
	\end{verbatim}
	} Some optimization solvers can perform parallel solution procedures, using more than a thread to tackle a problem by applying Branch-and-Cut, heuristics and maybe other solution strategies. The \texttt{threads} command is used to specify the maximum number of threads the solver is allowed to use, assuming the specified solver can use more than only one.
	
	\textbf{Usage example:} 
	\begin{verbatim}
		solve_lp --file DATA_FILE.lp --time 120 --threads 4
	\end{verbatim}

	{\color{mordantred19}
	\begin{verbatim}
		--modeling-language
	\end{verbatim}
	} Used to specify which modeling language will be used by \SolveLP among the ones available, which we describe in a later section. Each modeling language have its own pros and cons, and sometimes this will not have any impact on the overall solution process. Experienced users may find this useful to test/apply some particular aspect of his/her favorite modeling language.
	
	\textbf{Usage example:} 
	\begin{verbatim}
		solve_lp --file DATA.mat --modeling-language OR-TOOLS --solver CBC
	\end{verbatim}

	{\color{mordantred19}
	\begin{verbatim}
		--zeros/--no-zeros
	\end{verbatim}
	} The command \texttt{zeros} is used to display all the variables in the final result, even the ones that have zero value. In cases the expected result is mostly made of zero values, the user can use \texttt{no-zeros} to avoid an unnecessarily large result print.
	
	\textbf{Usage example:} 
	\begin{verbatim}
		solve_lp --file DATA_FILE.mat --time 300 --zeros
		solve_lp --file DATA_FILE.lp --solver SCIP --no-zeros
	\end{verbatim}

	{\color{mordantred19}
	\begin{verbatim}
		--log/--no-log
	\end{verbatim}
	} Optimization solvers commonly offer the option of printing a log of the solution procedure. The commands \texttt{log} and \texttt{no-log} are used to control if the user want to see this log or not.
	
	\textbf{Usage example:} 
	\begin{verbatim}
		solve_lp --file DATA_FILE.mat --time 900 --solver CBC --log
		solve_lp --file DATA_FILE.lp --modeling-language GUROBIPY --no-log
	\end{verbatim}

	{\color{mordantred19}
	\begin{verbatim}
		--verbose/--no-verbose
	\end{verbatim}
	} The pair of commands \texttt{verbose} and \texttt{no-verbose} are used to show or not to show the \SolveLP solving process steps log. We emphasize this is not the solver log, but a small log of the steps described by \SolveLP as it read the input data file, process it and solve the input problem.
	
	\textbf{Usage example:} 
	\begin{verbatim}
		solve_lp --file DATA_FILE.json --verbose --no-log
		solve_lp --file DATA_FILE.mps --time 30 --no-verbose --no-zeros
	\end{verbatim}


	\subsection{Terminology to Know}
			\begin{itemize}
				\item Preamble:\, All the code that precedes your \cs{begin}\brackets{document}. The preamble section of your document is where all the formatting takes place. 
				\item Commands:\, Commands are special words that determine \LaTeX behavior.
				\item Environments:\, Environments are used to format blocks of text in a \LaTeX documents
			\end{itemize}
	\subsection{Creating Templates}
		One of the many great things about \LaTeX\ is that offers many ways to avoid some of the repetitive things that come with creating several documents. One of the features \LaTeX\ offers is to create personal user templates for files that a user often uses but does not want to continuously type out every time.\\
		For example, lets say you wanted to save a general layout of a lesson document that included \textbf{both} the preamble and some of the actual text and commands in the document. 
		To create the template, one would first proceed to the folder originally containing this pdf (The Contents You Downloaded from the Zip File)) and then go to the folder labeled \textbf{Files to make Templates out of}. Once there, click the file named: \textbf{Lesson Template}. Once you have opened that file in texstudio, go to \textbf{File}, then \textbf{Make Template}. Once that is done, give the template a name and then press ok.\\
		After creating a template, all a user needs to do to use that template is to go back to \textbf{File}, and then \textbf{New From Template} and then click the template you created.
	\vspace{-1.5mm}
\newpage

%%%%%%%%%%%%%%%%%%%%%%%%%%%%%%%%%%%%%%%%%%%%%%%%%%%%%%%%%%%
\section{Using \SolveLP}
\vspace{10.5cm}
	\subsection{First Example}
		\SolveLP receives as input the matricial representation of a Linear Programming Problem, following the notations:

		\begin{itemize}
			\item $x$ represents the decision variables;
			\item $c$ represents the costs vector;
			\item $b$ represents the resources vector (a.k.a. the RHS - Right Hand Side - vector);
			\item $A$ represents the (...)
		\end{itemize}

		In which follows, it's considered as a linear programming problem in \emph{standard format} a minimization problem where all the constraints are equalities and all the variables are greater or equal to zero and do not have any upper bound. Using the formal mathematical notation, it can be generically represented as in \eqref{LP_STD}. 
		\begin{equation}\label{LP_STD}
			\begin{array}{llc}
				\min                & &  c^T x          \\
				\operatorname{s.t.} & &  Ax = b         \\
									& &   x \geqslant 0
			\end{array}
		\end{equation}

			\begin{equation}\label{LP_1_0_0}
				\begin{array}{llrcrcrcrcrcr}
				\min  & & 7x_1  & + & 9x_2  & + & 8x_3  & + & 5x_4  & + & 4x_5 &           &    \\
				\st   & & 9x_1  & + & 4x_2  & + & 6x_3  & + & 6x_4  & + & 5x_5 & =         & 42 \\
					  & & 4x_1  & + & 8x_2  & + & 3x_3  & + & 4x_4  & + & 1x_5 & =         & 62 \\
					  & & 8x_1  & + & 5x_2  & + & 5x_3  & + & 4x_4  & + & 7x_5 & =         & 46 \\
					  & &  x_1, &   &  x_2, &   &  x_3, &   &  x_4, &   &  x_5 & \geqslant & 0
				\end{array}
				\tag{LP1}
			\end{equation}

		It is not hard to solve \eqref{LP_1_0_0} to optimality and find that its optimal value is:
		$$x_1 = 0, x_2 = \frac{20}{3}, x_3 = 0, x_4 = 2, x_5 = \frac{2}{3}$$



%%%%%%%%%%%%%%%%%%%%%%%%%%%%%%%%%%%%%%%%%%%%%%%%%%%%%%%%%%%
\section{Command Line Interfaces (CLI's)}
\vspace{10.5cm}
	\subsection{Lesson and Exam File Structure}
			When using either the lesson or exam formats, use the following structure when creating your document:
			\begin{verbatim}
				\input{lesson}
				\begin{document}
				Text
				\end{document}
			\end{verbatim} 
	\subsection{Normal Classes}
			For situations when you do not want to use one of the pre-created/defined classes, you can begin your document with \cs{documentclass\brackets{"type of document"}}.
		\subsubsection{General Structure of Normal Documents}
			Usually most Latex Files will follow the following format:
			\begin{verbatim}
				\documentclass{class}
					\emph{Preamble}
				\begin{document}
					\emph{Text}
				\end{document}
			\end{verbatim} 
%%%%%%%%%%%%%%%%%%%%%%%%%%%%%%%%%%%%%%%%%%%%%%%%%%%%%%%%%%%
\section{Using Solve\_LP}
\vspace{10.5cm}

\section{List of Already Defined Environments and Macros}	
\vspace{10.5cm}
The following subsections are lists and examples of pre-defined macros or commands. Creating a macro in latex is effectively similar to creating a shortcut. Macros allow for a cleaner, more efficient process of writing the code for your document. \\
Macros are great for repetitive elements and texts that you carry across several documents.
	\subsection{Main Box Environments}
		\subsubsection{Theorems}
			\begin{docEnvironment}{theo}{}
				This environment \cs{begin\brackets{Theo}} inserts a new yellow theorem box with a black frame into the document. 
			\end{docEnvironment}
		\subsubsection{Boxes for Definitions}
			\begin{docEnvironment}{definition}{}
				This environment \cs{begin\brackets{definition}} inserts a regular definition into the document. 
			\end{docEnvironment}
		\subsubsection{Boxes for Examples}
			\begin{docEnvironment}{example}{\brackets{\sl{title}}}
				This environment \cs{begin\brackets{example}} creates a regular example box. \\
				After inserting the question for the specified example, make sure to use the command \cs{tcbline} or \cs{tcblower}. This command creates a dashed line within the example box or any pre-created box and allows for you to created a specified space for the students to fill in the solution. 
			\end{docEnvironment}
			\begin{docEnvironment}{longexample}{\brackets{\sl{title}}}
				Environment for creating examples inside lessons whose text is quite lengthy or the example requires a good amount of work or drawing. \\
				If one wants to extend the box of the height further that the default value, go to the lesson.tex file inside of the lessons folder. Once inside of the lesson.tex file, go down to about line 136, inside the examstyle./style, and change the height from 9.5cm to your preferred value. 
			\end{docEnvironment}
		\subsubsection{Boxes for Discussions}
			\begin{docEnvironment}{discussion}{\brackets{\sl{title}}}
				This environment inserts a box similar to the example box just instead labeled discussion. \\ 
				Similar to the example box as well, after inserting the question or text, make sure to use the command \cs{tcbline}. This command will create a dashed line within the box and allows for you to created a specified space for the students to fill in the solution or drawing etc. preferred value. 
			\end{docEnvironment}
		\subsubsection{Boxes for Visualizations}
			\begin{docEnvironment}{visualization}{\brackets{\sl{title}}}
				This environment inserts a box similar to the example and discussion boxes just instead for visualizations. 
			\end{docEnvironment}
	\subsection{Extra Box Environment}
		\begin{docEnvironment}{DashedDefinition}{}
			This command offers another option for a definition box just with a dashed frame.
		\end{docEnvironment}
\vspace{2mm}		
\newgeometry{
	left=14mm, 
	right=14mm, 
	top=9mm, 
	bottom=13mm}
%%%%%%%%%%%%%%%%%%%%%%%%%%%%%%%%%%%
% ------- Code & Examples ------- %
%%%%%%%%%%%%%%%%%%%%%%%%%%%%%%%%%%%
\begin{disexam}
\begin{DashedDefinition}{}
[A partial derivative of a function of several variables is its derivative with respect to one of those variables, with the others held constant. Partial derivatives are used in vector calculus and differential geometry.
\end{DashedDefinition}
\end{disexam}
\vspace*{0.75\baselineskip}
%===============================
\begin{disexam}
\vspace{1mm}
\begin{visualization}[\quad \large Angular Momentum \hspace{3mm}]
Discuss an alternate way of using the conservation of angular momentum for satellite orbits and any other point masses moving in a circle.
\tcbline
\end{visualization}
\vspace{1mm}
\end{disexam}
\vspace*{0.75\baselineskip}
%===============================
\begin{disexam}
\begin{example2}
\vspace{1mm}
A partial derivative of a function of several variables is its derivative with respect to one of those variables, with the others held constant. Partial derivatives are used in vector calculus and differential geometry.
\tcbline
\end{example2}
\end{disexam}
%===============================
	\vspace*{1\baselineskip}
\begin{disexam}
\begin{example}[\quad \large Rotational Motion]
Now assume the small pulley has rotational inertia.
\begin{enumerate}
\item Will your answers be different? Why?
\item How does the angular velocity of the rotating apparatus and the linear velocity of the falling mass compare to the previous case?
\item How does the torque exerted by the string on the rotating apparatus change the angular momentum of the apparatus?
\end{enumerate}
\tcbline
\vspace{1mm}
\solution
\end{example}
\end{disexam}
%===============================
\begin{disexam}
\begin{longexample}[\quad \large Rotational Motion]
A block of mass $4m$ is attached to a light string and passes over a pulley with negligible rotational inertia and is wrapped around a vertical pole of radius $r$.  The block is then released from rest, causing the string to unwind and the vertical pole it is wrapped around to rotate.  On top of the vertical pole lies a horizontal rod of length $2L$ with block2s of mass $m$ attached to both ends.  The rotational inertia of this apparatus is $2mL^2$.
\begin{enumerate}
\item What is the tension in the string in terms of the acceleration of the falling block?
\item What is the torque exerted on the rotating pole by the string in terms of the acceleration of the falling block?
\item When the large block has fallen a distance $D$, what is the instantaneous rotational kinetic energy of the apparatus?
\end{enumerate}
\tcblower
\vspace{1mm}
\solution
\end{longexample}
\end{disexam}
%===============================
\newgeometry{
	left=29mm, 
	right=29mm, 
	top=20mm, 
	bottom=15mm}
%===============================
\vspace{1.5mm}
	\subsection{Extra Box Commands}
		\begin{docCommand}{tcbline or tcblower}{}
				Creates a dashed line within the box. Useful for creating a specified portion for students to write their answer down.
		\end{docCommand}
		\begin{docCommand}{solution}{}
				Inserts the word solution in red font.
		\end{docCommand}
	\subsection{Commands (Macros)}
		\begin{docCommand}{contactinfo}{}
				Allows for a quicker way to input contact information through several documents.\\ 
				To insert contact information, put the following in your preamble:\\ \cs{newcommand}\brackets{\cs{contactinfo}}\brackets{Insert info here.}
		\end{docCommand}
		\begin{docCommand}{fillin}{}
			Allows for a quicker way to input a line for students to fill in the notes.
		\end{docCommand}
	\subsection{How to Define New Commands (Macros) and Colors}
		\begin{docCommand}{newcommand}{\brackets{\cs{name}}\brackets{\cs{action}}}
			The command \cs{newcommand} allows the user to effectively use a shortcut for a given action. Allows for a cleaner, more organized code. When creating a new command, make sure to put it in the preamble of you document.  
		\end{docCommand}
		\begin{docCommand}{definecolor}{\brackets{name for color}\brackets{rgb}}
			The \cs{definecolor} command allows for the user to define new colors to be used in the document. For the rgb values of different colors, go to latexcolor.com.
		\end{docCommand}
\newpage
%%%%%%%%%%%%%%%%%%%%%%%%%%%%%%%%%%%%%%%%%%%%%%%%%%%%%%%%%%%

\section{Font Styles}
\vspace{10.5cm}
	\subsection{Font Types}
\[\begin{array}{ccc}
\cs{textit\brackets{...}} & \textit{italic} & \textup{Italic shape, used mostly for emphasis.}\\
\cs{textsl\brackets{...}} & \textsl{slanted} & \textup{Slanted shape, a bit different from italic.}\\
\cs{textsc\brackets{...}} & \textsc{Small Caps} & \textup{Small caps shape, use sparingly.}\\
\cs{textup\brackets{...}} & \textup{upright} & \textup{Upright shape, usually the default.}\\
\cs{textbf\brackets{...}} & \textbf{boldface} & \textup{Boldface series, often used for headings.}\\
\cs{textmd\brackets{...}} & \textmd{medium} & \textup{Medium series, usually the default.}\\
\cs{textrm\brackets{...}} & \textrm{roman} & \textup{Roman family, usually the default.}\\
\cs{textsf\brackets{...}} & \textsf{sans serif} & \textup{Sans Serif family, used for posters, etc.}\\
\cs{texttt\brackets{...}} & \texttt{typewriter} & \textup{Typewriter family, fixed-pitch characters.}\\
\cs{emph\brackets{...}} & \emph{emphasized} & \textup{Use for emphasis, usually changes to italic.}\\		
\end{array}\]
\newpage
\newgeometry{
	left=32mm, right=18mm, top=20mm, bottom=18mm,
	marginparwidth=28mm, marginparsep=4mm}
%%%%%%%%%%%%%%%%%%%%%%%%%%%%%%%%%%%%%%%%%%%%%%%%%%%%%%%%%%%
\section{Referencing and Documentation}
\vspace{10.5cm}
	\subsection{References/Hyperlinks and Formatting}
		\begin{docCommand}{label}{\brackets{\sl{link}}}
			In order to refer to make a cross-reference within a document, you first must label the text/section that you want to refer to.
		\end{docCommand}
		\begin{verbatim}
	            \subsection{Math Tests/Exams and Math Symbols/Functions}\label{mathexams}
		\end{verbatim}
		\begin{docCommand}{ref}{\brackets{\sl{label-name}}}
			This command allows you to make a cross-reference to another part of your document. 
		\end{docCommand}
		\begin{docCommand}{href}{\brackets{\sl{label-name}}}
			This command is used when the user wants to insert a link that is able to clicked on into a document but want's there to be a placeholder to cover the hyperlink. An example is below. 
		\end{docCommand}
\begin{disexam}
\href{mailto:armindubert19@gmail.com}{basis independent mclean}
\end{disexam}
		\begin{docCommand}{fullref}{\brackets{\sl{label-name}}}
			This command allows you to make a cross-reference to another part of your document. The \cs{fullref} includes both the section and page number that you are referring to as opposed to the command \cs{ref} that only refers to the section.
		\end{docCommand}
		\begin{docCommand}{url}{\brackets{\sl{link}}}
			This command allows you to insert a link into your document that's able to be clicked on in the pdf you create.
		\end{docCommand}
		\begin{docCommand}{hypersetup}{\brackets{colorlinks,linkcolor={red!50!black},citecolor={blue!50!black}, urlcolor={blue!80!black}}}
			This following command makes the links you include in your code/file more appealing by removing the box around the links as well as color coding the different links. This command should be inserted in the preamble of your document.
		\end{docCommand}
	\subsection{Documentation}
	Documentation and Listings are commands and ways of showing source code side by side with the behavior of the command.\\
	To use documentation in latex, make sure to input \cs{tcbuselibrary}\brackets{listings}.
		\begin{docCommand}{cs}{}
			Allows you to typeset a command without having the command actually played out.
		\end{docCommand}
		\begin{disexam}
\cs{marg}
		\end{disexam}
		\begin{docCommand}{brackets}{}
			Allows you to put brackets around text that will appear in the document.
		\end{docCommand}
		\begin{disexam}
\cs{brackets}
		\end{disexam}
		\begin{docEnvironment}{docCommand}{}
			Allows you to identify a command and define what it does, similar to all of the lines with a yellow background in this pdf.
		\end{docEnvironment}
		For more on documentation, see tcolorbox.pdf
	\subsection{Verbatim}
		To reproduce new lines and spaces exactly as they are in your input file, use the verbatim environment. The verbatim environment prints its text in typewriter-style type and sets it off from the rest of the document with blank lines before and after.
		\begin{docEnvironment}{verbatim}{}
			Allows you to identify a command and define what it does, similar to all of the lines with a yellow background in this pdf.
		\end{docEnvironment}
\newpage
%%%%%%%%%%%%%%%%%%%%%%%%%%%%%%%%%%%%%%%

	\begin{thebibliography}{17}
	\vspace{10.5cm}
		\bibitem{RPI} Academic and Research Computing. \textit{Text Formatting with \LaTeX\ A Tutorial}. NY, April 2007.\\
				\url{http://www.rpi.edu/dept/arc/docs/latex/latex-intro.pdf}
		\bibitem{is.skills} Leslie Lamport. \textit{\LaTeX\ for Beginners}.  fifth edition, Document Reference: 3722-2014, March 2014.\\
				\url{http://www.docs.is.ed.ac.uk/skills/documents/3722/3722-2014.pdf}
		\bibitem{Tobias Oetiker} Tobias Oetiker, Hubert Partl, Irene Hyna, and Elisabeth Schlegl. \textit{The Not So Short Introduction to \LaTeX\ }. Version 6.3, March 1994.\\
				\url{https://tobi.oetiker.ch/lshort/lshort.pdf}
		\bibitem{Helmut Kopka} Helmut Kopka and Patrick W. Daly. \textit{A Guide to \LaTeX\ and Electronic Publishing}. Addison-Wesley, fourth edition, May 2003.\\
				\url{https://www2.mps.mpg.de/homes/daly/GTL/gtl_20030512.pdf}
		\bibitem{Philip Hirschhorn} Philip Hirschhorn. \textit{Using the exam document class}. Wellesley College, second edition, MA, November 2017.\\
				\url{http://www-math.mit.edu/~psh/exam/examdoc.pdf}
		\bibitem{web1} \textit{When should I use \cs{input} vs \cs{include}}.\\
				\url{https://tex.stackexchange.com/questions/246/when-should-i-use-input-vs-include}		
		\bibitem{Overleaf} Overleaf. \textit{Inserting Images}.\\
				\url{https://www.overleaf.com/learn/latex/Inserting_Images}
		\bibitem{Rice} Rice University. \textit{\LaTeX\ Mathematical Symbols}.\\
			\url{https://www.caam.rice.edu/~heinken/latex/symbols.pdf}
		\bibitem{Overleaf1} Overleaf. \textit{Integrals, sum and limits}.\\
			\url{https://www.overleaf.com/learn/latex/Integrals,_sums_and_limits}
		\bibitem{BU} Boston University. \textit{\LaTeX\ Command Summary}. December 1994.\\
			\url{https://www.bu.edu/math/files/2013/08/LongTeX1.pdf}
		\bibitem{Overleaf2} Overleaf. \textit{Subscripts and superscripts}.\\
			\url{https://www.overleaf.com/learn/latex/Subscripts_and_superscripts}
		\bibitem{Disquis} Disquis. \textit{\LaTeX\ Color}. Addison-Wesley, second edition, Reading, MA, 1994.\\
				\url{http://latexcolor.com/}
		\bibitem{David Woods} David Woods.\textit{Useful \LaTeX\ Commands}.\\
				\url{https://www.scss.tcd.ie/~dwoods/1617/CS1LL2/HT/wk1/commands.pdf}
		\bibitem{Overleaf3} Overleaf. \textit{Spacing in Math Mode}.\\
				\url{https://www.overleaf.com/learn/latex/Spacing_in_math_mode}
		\bibitem{Overleaf4} Overleaf. \textit{Fractions and Binomials}.\\
				\url{https://www.overleaf.com/learn/latex/Fractions_and_Binomials}
		\bibitem{Overleaf5} Overleaf. \textit{Environments}.\\
				\url{https://www.overleaf.com/learn/latex/Environments}
		\bibitem{Overleaf6} Overleaf. \textit{Margin Notes}.\\
				\url{https://www.overleaf.com/learn/latex/Margin_notes}
		\bibitem{Math} Art of Problem Solving. \textit{LaTeX:Commands}\\
				\url{https://artofproblemsolving.com/wiki/index.php/LaTeX:Commands}
		\bibitem{LaTeX} Latex-Project.\\
				\url{https://www.latex-project.org/about/}
		\bibitem{texstackexchange} Tex Stack Exchange. customizing part style with Tikz.\\
				\url{https://tex.stackexchange.com/questions/159551/customizing-part-style-with-tikz}
		\bibitem{texstackexchange2}	Tex Stack Exchange. How to change chapter/section style in tufte-book?.\\
				\url{https://tex.stackexchange.com/questions/83057/how-to-change-chapter-section-style-in-tufte-book?noredirect=1&lq=1}
	\end{thebibliography}
\addtocounter{section}{14}
\addcontentsline{toc}{section}{\protect\numberline{\thesection}~~~ References}
%%%%%%%%%%%%%%%%%%%%%%%%%%%%%%%%%%%%%%%%%%%%%%%%%%%%%%%%%%%%%%%%%%
\end{document}